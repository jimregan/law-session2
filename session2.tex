\documentclass{beamer} %[compress, blue]
\mode<presentation>

\setbeamercolor{frametitle}{fg=lightred,bg=darkred}
\setbeamercolor{title}{fg=lightred,bg=darkred}
\usepackage[spanish,polutonikogreek,english]{babel}
\usepackage{ucs}
\usepackage{times}
\usepackage[T1]{fontenc}
\usepackage{tipa} %ŋ character
\usepackage[utf8x]{inputenc}
\usepackage[absolute,overlay]{textpos}
\usepackage{graphicx}
%\usepackage[small,bf]{caption}
%\usepackage{tabularx}
\usepackage{tikz}
\usepackage{url}
\usepackage{gb4e}
\usepackage{linguex}
\usepackage{cgloss4e}
\usepackage{multirow}
\usepackage{ragged2e} 
\usepackage{wasysym}
\usepackage{alltt}

\usetheme{Warsaw}

\definecolor{darkred}{RGB}{153,0,0}
\definecolor{lightred}{RGB}{226,200,200}
\definecolor{ranis}{RGB}{ 128,128,128}

\newcommand\cyrtext[1]{{\fontencoding{T2A}\selectfont #1}}
\newcommand\grktext[1]{{\fontencoding{LGR}\fontfamily{pmt}\selectfont #1}}

\useoutertheme[subsection=false]{smoothbars}

%\usecolortheme{seahorse} %beetle, albatross, fly, default light, seahorse, crane, dove
\usecolortheme[named= darkred]{structure}

\setbeamertemplate{footline}[text line]{} % makes the footer EMPTY
\setbeamertemplate{headline}[text line]{} % header empty

\setbeamersize{text margin left=0.5cm}
\setbeamertemplate{navigation symbols}{}


%\logo{\includegraphics[height=1.6cm]{logoLAW.png}}
\pgfdeclareimage[height=1.6cm]{logo}{logoLAWnotrans.png}

\setlength{\TPHorizModule}{1mm}
\setlength{\TPVertModule}{1mm}

\newcommand{\MyLogoCentred}{
\begin{textblock}{14}(57.2,10.5)
  \pgfuseimage{logo}
\end{textblock}
}

\newcommand{\MyLogoBottomCentred}{
\begin{textblock}{14}(53.5,70)
  \pgfuseimage{logo}
\end{textblock}
}

\newcommand{\MyLogoBottomRight}{
\begin{textblock}{14}(112.2,80.0)
  \pgfuseimage{logo}
\end{textblock}
}

\newcommand{\MyLogo}{
\begin{textblock}{14}(112.2,0.5)
  \pgfuseimage{logo}
\end{textblock}
}

\date{3rd May 2011}
\title{Session 2: Advanced Dictionaries}

\author{Jimmy O'Regan}

\begin{document}


\frame{\titlepage \MyLogoBottomCentred}


\begin{frame}
  \frametitle{Advanced Dictionaries: Table of Contents}
  \tableofcontents
\end{frame}


\section{Recap}

\begin{frame}
  \frametitle{Recap}
  Yesterday we saw the general layout of Apertium's dictionaries, and the layout of
  individual entries in particular.
\end{frame}

\begin{frame}
  \frametitle{Dictionary overview}

  \begin{exampleblock}{What a dictionary looks like:}
    \begin{footnotesize}
    \begin{alltt}
      <?xml version="1.0" encoding="UTF-8"?>\\
      <dictionary>\\
      ~<alphabet></alphabet>\\
      ~<sdefs>\\
      ~~<sdef n="n" c="noun"/>\\
      ~~...\\
      ~</sdefs>\\
      ~<pardefs>\\
      ~~<pardef n="example\_\_n">\\
      ~~~...\\
      ~~</pardef>\\
      ~</pardefs>\\
      ~<section id="id" type="standard">\\
      ~~...\\
      ~</section>\\
      </dictionary>\\
    \end{alltt}
    \end{footnotesize}
\end{exampleblock}
\end{frame}

\begin{frame}
  \frametitle{Entry}

  \begin{exampleblock}{Simple entries:}
    \begin{footnotesize}
    \begin{alltt}
      <e> \\
      ~~<p> \\
      ~~~~<l>beer</l> \\
      ~~~~<r>beer<s n="n"/><s n="sg"/></r> \\
      ~~</p> \\
      </e> \\
      <e> \\
      ~~<p> \\
      ~~~~<l>beers</l> \\
      ~~~~<r>beer<s n="n"/><s n="pl"/></r> \\
      ~~</p> \\
      </e>
    \end{alltt}
    \end{footnotesize}
\end{exampleblock}
\end{frame}

\begin{frame}
  \frametitle{Paradigms}

  \begin{exampleblock}{Paradigms contain common parts:}
    \begin{footnotesize}
    \begin{alltt}
      <pardef n="beer\_\_n"> \\
      ~~<e> \\
      ~~~~<p> \\
      ~~~~~~<l></l> \\
      ~~~~~~<r><s n="n"/><s n="sg"/></r> \\
      ~~~~</p> \\
      ~~</e> \\
      ~~<e> \\
      ~~~~<p> \\
      ~~~~~~<l>s</l> \\
      ~~~~~~<r><s n="n"/><s n="pl"/></r> \\
      ~~~~</p> \\
      ~~</e> \\
      </pardef>
    \end{alltt}
    \end{footnotesize}
\end{exampleblock}
\end{frame}

\begin{frame}
  \frametitle{Paradigms can appear anywhere}

  \begin{exampleblock}{Paradigms are also used to normalise}
    \begin{footnotesize}
    \begin{alltt}
      <pardef n="o\_ou"> \\
      ~~<e> \\
      ~~~~<p> \\
      ~~~~~~<l>ou</l> \\
      ~~~~~~<r>ou</r> \\
      ~~~~</p> \\
      ~~</e> \\
      ~~<e r="LR"> \\
      ~~~~<p> \\
      ~~~~~~<l>o</l> \\
      ~~~~~~<r>ou</r> \\
      ~~~~</p> \\
      ~~</e> \\
      </pardef>
    \end{alltt}
    \end{footnotesize}
\end{exampleblock}

  This paradigm allows us to accept American spellings (color/colour), while using 
  British spellings as the default.
\end{frame}

\section{eXtensible}

\begin{frame}
  \frametitle{XML: eXtensible}

  XML is designed to be extended. We can use standard tools, available with
  every modern operating system, to extend the language to new situations:

  \begin{exampleblock}{Paradigms are also used to normalise}
    \begin{footnotesize}
    \begin{alltt}
      <pardef n="o\_ou"> \\
      ~~<e v="en\_GB"> \\
      ~~~~<p> \\
      ~~~~~~<l>ou</l> \\
      ~~~~~~<r>ou</r> \\
      ~~~~</p> \\
      ~~</e> \\
      ~~<e v="en\_US"> \\
      ~~~~<p> \\
      ~~~~~~<l>o</l> \\
      ~~~~~~<r>ou</r> \\
      ~~~~</p> \\
      ~~</e> \\
      </pardef>
    \end{alltt}
    \end{footnotesize}
\end{exampleblock}
\end{frame}
\section{V attribute}
\begin{frame}
  \frametitle{{\it V}ariant}
    The {\it v} (variant) attribute is not part of the standard dictionary definition. It is processed
    with XSLT so we can generate British and American variants from the same dictionary.
\end{frame}

\section{Words vs. Translation units}
\begin{frame}
  \frametitle{Words vs. Translation units}
  Many translation systems take the simplistic view that a word is the unit of translation.

  This is not always true.
\end{frame}

\section{Enclitics}
\begin{frame}
  \frametitle{Enclitics}
  Many of the Romance languages that Apertium was originally designed for use {\it enclitic pronouns}:

  d\'imelo = decir_{{\it Verb, Imperative}} $+$ me_{{\it Prn}} $+$ lo_{{\it Prn}}
\end{frame}

\section{The j tag}
\begin{frame}
  \frametitle{The J tag}
  The {\tt $<$j/$>$} tag tells the analyser that it is to analyse the input as {\it joined} words:
  \begin{exampleblock}{Paradigms are also used to normalise}
    \begin{footnotesize}
    \begin{alltt}
      ~~<e> \\
      ~~~~<p> \\
      ~~~~~~<l>dímelo</l> \\
      ~~~~~~<r>decir<s n="vblex"><j/>me<s n="prn"/><j/>lo<s n="prn"/></r> \\
      ~~~~</p> \\
      ~~</e> \\
    \end{alltt}
    \end{footnotesize}
\end{exampleblock}
\end{frame}

\end{document}
