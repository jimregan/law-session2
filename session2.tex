\documentclass{beamer} %[compress, blue]
\mode<presentation>

\setbeamercolor{frametitle}{fg=lightred,bg=darkred}
\setbeamercolor{title}{fg=lightred,bg=darkred}
\usepackage[spanish,polutonikogreek,english]{babel}
\usepackage{ucs}
\usepackage{times}
\usepackage[T1]{fontenc}
\usepackage{tipa} %ŋ character
\usepackage[utf8x]{inputenc}
\usepackage[absolute,overlay]{textpos}
\usepackage{graphicx}
%\usepackage[small,bf]{caption}
%\usepackage{tabularx}
\usepackage{tikz}
\usepackage{url}
\usepackage{gb4e}
\usepackage{linguex}
\usepackage{cgloss4e}
\usepackage{multirow}
\usepackage{ragged2e} 
\usepackage{wasysym}
\usepackage{alltt}
\usepackage{epigraph}

\usetheme{Warsaw}

\definecolor{darkred}{RGB}{153,0,0}
\definecolor{lightred}{RGB}{226,200,200}
\definecolor{ranis}{RGB}{ 128,128,128}

\newcommand\cyrtext[1]{{\fontencoding{T2A}\selectfont #1}}
\newcommand\grktext[1]{{\fontencoding{LGR}\fontfamily{pmt}\selectfont #1}}

\useoutertheme[subsection=false]{smoothbars}

%\usecolortheme{seahorse} %beetle, albatross, fly, default light, seahorse, crane, dove
\usecolortheme[named= darkred]{structure}

\setbeamertemplate{footline}[text line]{} % makes the footer EMPTY
\setbeamertemplate{headline}[text line]{} % header empty

\setbeamersize{text margin left=0.5cm}
\setbeamertemplate{navigation symbols}{}


%\logo{\includegraphics[height=1.6cm]{logoLAW.png}}
\pgfdeclareimage[height=1.6cm]{logo}{logoLAWnotrans.png}

\setlength{\TPHorizModule}{1mm}
\setlength{\TPVertModule}{1mm}

\newcommand{\MyLogoCentred}{
\begin{textblock}{14}(57.2,10.5)
  \pgfuseimage{logo}
\end{textblock}
}

\newcommand{\MyLogoBottomCentred}{
\begin{textblock}{14}(53.5,70)
  \pgfuseimage{logo}
\end{textblock}
}

\newcommand{\MyLogoBottomRight}{
\begin{textblock}{14}(112.2,80.0)
  \pgfuseimage{logo}
\end{textblock}
}

\newcommand{\MyLogo}{
\begin{textblock}{14}(112.2,0.5)
  \pgfuseimage{logo}
\end{textblock}
}

\date{3rd May 2011}
\title{Session 2: Advanced Dictionaries}

\author{Jimmy O'Regan}

\begin{document}


\frame{\titlepage \MyLogoBottomCentred}


\begin{frame}
  \frametitle{Advanced Dictionaries: Table of Contents}
  \tableofcontents
\end{frame}


\section{Recap}

\begin{frame}
  \frametitle{Recap}
  Yesterday we saw the general layout of Apertium's dictionaries, and the layout of
  individual entries in particular.
\end{frame}

\begin{frame}
  \frametitle{Dictionary overview}

  \begin{exampleblock}{What a dictionary looks like:}
    \begin{footnotesize}
    \begin{alltt}
      <?xml version="1.0" encoding="UTF-8"?>\\
      <dictionary>\\
      ~<alphabet></alphabet>\\
      ~<sdefs>\\
      ~~<sdef n="n" c="noun"/>\\
      ~~...\\
      ~</sdefs>\\
      ~<pardefs>\\
      ~~<pardef n="example\_\_n">\\
      ~~~...\\
      ~~</pardef>\\
      ~</pardefs>\\
      ~<section id="id" type="standard">\\
      ~~...\\
      ~</section>\\
      </dictionary>\\
    \end{alltt}
    \end{footnotesize}
\end{exampleblock}
\end{frame}

\begin{frame}
  \frametitle{Entry}

  \begin{exampleblock}{Simple entries:}
    \begin{footnotesize}
    \begin{alltt}
      <e> \\
      ~~<p> \\
      ~~~~<l>beer</l> \\
      ~~~~<r>beer<s n="n"/><s n="sg"/></r> \\
      ~~</p> \\
      </e> \\
      <e> \\
      ~~<p> \\
      ~~~~<l>beers</l> \\
      ~~~~<r>beer<s n="n"/><s n="pl"/></r> \\
      ~~</p> \\
      </e>
    \end{alltt}
    \end{footnotesize}
\end{exampleblock}
\end{frame}

\begin{frame}
  \frametitle{Paradigms}

  \begin{exampleblock}{Paradigms contain common parts:}
    \begin{footnotesize}
    \begin{alltt}
      <pardef n="beer\_\_n"> \\
      ~~<e> \\
      ~~~~<p> \\
      ~~~~~~<l></l> \\
      ~~~~~~<r><s n="n"/><s n="sg"/></r> \\
      ~~~~</p> \\
      ~~</e> \\
      ~~<e> \\
      ~~~~<p> \\
      ~~~~~~<l>s</l> \\
      ~~~~~~<r><s n="n"/><s n="pl"/></r> \\
      ~~~~</p> \\
      ~~</e> \\
      </pardef>
    \end{alltt}
    \end{footnotesize}
\end{exampleblock}
\end{frame}

\begin{frame}
  \frametitle{Paradigms can appear anywhere}

  \begin{exampleblock}{Paradigms are also used to normalise}
    \begin{footnotesize}
    \begin{alltt}
      <pardef n="o\_ou"> \\
      ~~<e> \\
      ~~~~<p> \\
      ~~~~~~<l>ou</l> \\
      ~~~~~~<r>ou</r> \\
      ~~~~</p> \\
      ~~</e> \\
      ~~<e r="LR"> \\
      ~~~~<p> \\
      ~~~~~~<l>o</l> \\
      ~~~~~~<r>ou</r> \\
      ~~~~</p> \\
      ~~</e> \\
      </pardef>
    \end{alltt}
    \end{footnotesize}
\end{exampleblock}

  This paradigm allows us to accept American spellings (color/colour), while using 
  British spellings as the default.
\end{frame}

\section{eXtensible}

\begin{frame}
  \frametitle{XML: eXtensible}

  XML is designed to be extended. We can use standard tools, available with
  every modern operating system, to extend the language to new situations:

  \begin{exampleblock}{British vs. American spelling}
    \begin{footnotesize}
    \begin{alltt}
      <pardef n="o\_ou"> \\
      ~~<e v="en\_GB"> \\
      ~~~~<p> \\
      ~~~~~~<l>ou</l> \\
      ~~~~~~<r>ou</r> \\
      ~~~~</p> \\
      ~~</e> \\
      ~~<e v="en\_US"> \\
      ~~~~<p> \\
      ~~~~~~<l>o</l> \\
      ~~~~~~<r>ou</r> \\
      ~~~~</p> \\
      ~~</e> \\
      </pardef>
    \end{alltt}
    \end{footnotesize}
\end{exampleblock}
\end{frame}

\begin{frame}
  \frametitle{{\it V}ariant}
    The {\it v} (variant) attribute is not part of the standard dictionary definition. It is processed
    with XSLT so we can generate British and American variants from the same dictionary.
\end{frame}

\section{Words vs. Translation units}
\begin{frame}
  \frametitle{Words vs. Translation units}
  Many translation systems take the simplistic view that a word is the unit of translation.

  This is not always true.
\end{frame}

\begin{frame}
  \frametitle{Enclitics}
  Many of the Romance languages that Apertium was originally designed for use {\it enclitic pronouns}:

  d\'imelo = decir_{{\it Verb, Imperative}} $+$ me_{{\it Prn}} $+$ lo_{{\it Prn}}
\end{frame}

\begin{frame}
  \frametitle{The J tag}
  The {\tt $<$j/$>$} tag tells the analyser that it is to analyse the input as {\it joined} words:
  \begin{exampleblock}{Joined words}
    \begin{footnotesize}
    \begin{alltt}
      <e><p> \\
      ~<l>dímelo</l> \\
      ~<r>decir<s n="vblex"><j/>me<s n="prn"/><j/>lo<s n="prn"/></r> \\
      </p></e> \\
    \end{alltt}
    \end{footnotesize}
\end{exampleblock}
\end{frame}

\begin{frame}
  \frametitle{Multiwords}
  Just as a single "word" may in fact be composed of joined words, 
  a sequence of words may have a unique meaning as a unit.
  More importantly, a multiword unit may have a translation different
  to the translations of its parts: a "petty officer" is not petty in
  the usual sense.

  \pause

  A petty officer is not {\it necessarily} petty.

\end{frame}
\begin{frame}
  \frametitle{Tokenisation /1}
  I briefly mentioned tokenisation yesterday. Splitting a string of text into tokens
  is a fundamental first step in any language processing application.

  Most applications do this the same way, by treating spaces and punctuation as
  special characters, and splitting the text when they are encountered.

  \pause

   Apertium does almost exactly the opposite.
\end{frame}
\begin{frame}
  \frametitle{Tokenisation /2}
  In Apertium, the characters that make up a word are special: the 
  {\tt $<$alphabet$>$} tag containst them.

  As long as a word is contained in the dictionary, Apertium continues to 
  read it. Otherwise, if the characters are in the alphabet, it continues
  to read, and tokenises as soon as it sees a character not contained in
  the alphabet.

  \pause

  Spaces are not special, and can be part of any word.
\end{frame}
\begin{frame}
  \frametitle{Words to multiwords /1}

  \begin{exampleblock}{Recap: a single entry}
    \begin{footnotesize}
    \begin{alltt}
      ~~<e> \\
      ~~~~<i>officer</i> \\
      ~~~~<par n="house\_\_n> \\
      ~~</e> \\
    \end{alltt}
    \end{footnotesize}
  \end{exampleblock}

  (Recall that {\tt $<$i$>$} is the same as {\tt $<$p$>$}
  where {\tt $<$l$>$} and {\tt $<$r$>$} are identical.)
\end{frame}
\begin{frame}
  \frametitle{Words to multiwords /2}

  \begin{exampleblock}{A multiword entry}
    \begin{footnotesize}
    \begin{alltt}
      ~~<e> \\
      ~~~~<i>petty<b/>officer</i> \\
      ~~~~<par n="house\_\_n> \\
      ~~</e> \\
    \end{alltt}
    \end{footnotesize}
  \end{exampleblock}
\end{frame}
\begin{frame}
  \frametitle{The b tag}
  The {\tt $<$b/$>$} tag represents a {\it blank}.

  It typically represents a space, but can also represent a {\it superblank}:
  the internal formatting used in a document that might otherwise interfere with
  processing.
\end{frame}
\begin{frame}
  \frametitle{Inner inflection}
  Some multiwords may have {\it inner inflection}: the plural of "act of God" is
  "acts of God".

  \pause
  \begin{exampleblock}{A multiword entry}
    \begin{footnotesize}
    \begin{alltt}
      <e> \\
      ~<i>act</i> \\
      ~<par n="house\_\_n> \\
      ~<p> \\
      ~~<l><b/>of<b/>God</l> \\
      ~~<r><g><b/>of<b/>God</g></r> \\
      ~</p> \\
      </e> \\
    \end{alltt}
    \end{footnotesize}
  \end{exampleblock}
\end{frame}
\begin{frame}
  \frametitle{The g tag}
  The {\tt $<$g$>$} tag represents a {\it group} of words, that
  follow a word that inflects.

  The words within the group do not change.
\end{frame}

\section{Regular Expressions}
\begin{frame}
  \frametitle{Regular Expressions}
  \epigraph{Some people, when confronted with a problem, think "I know, I'll use regular expressions." Now they have two problems.}{Jamie Zawinski}
\end{frame}
\begin{frame}
  \frametitle{Regular Expressions}
  Regular expressions are a difficult, but powerful tool.

  Usually in Apertium their use is confined to entities that are not words, but which appear in documents and {\it usually}
  can be passed through unchanged -- numbers, web addresses, etc.
\end{frame}
\begin{frame}
  \frametitle{Example regular expression}

  \begin{exampleblock}{An example}
    \begin{footnotesize}
    \begin{alltt}
      <e> \\
      ~<re>[0-9]+</re> \\
      ~<p> \\
      ~~<l></l> \\
      ~~<r><s n="num"/></r> \\
      ~</p> \\
      </e> \\
    \end{alltt}
    \end{footnotesize}
  \end{exampleblock}
\begin{frame}
\end{frame}
  \frametitle{The regular expression, broken down}

  {\tt $[$0-9$]$} means "any digit from 0 to 9"

  {\tt $+$} means "match one or more"
  
  The {\tt $<$p$>$} tag attaches the symbol {\tt num} (number) to 
  the output.
\end{frame}


\end{document}
